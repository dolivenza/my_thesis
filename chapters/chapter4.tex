\chapter{APPLICATION OF PTR-MS IN BREATH ANALYSIS: KETONES}
\markboth{KETONES IN PTR-MS}{}


\section{Introduction}
Ketones are present in breath, faeces, urine, etc.
They are an indicator of the metabolism...


\section{Methodology}
In collaboration with IONICON Analytik GmbH (Innsbruck, Austria) and the Institute for Breath Analysis of the University of Innsbruck (Innsbruck, Austria) we measured 19 ketones, which are of interest for breath analysis. These are listed in table \ref{tb:k} and their structure is shown in figure \ref{fig:k}. We did it in dry and humid conditions and using both H$_3$O$^+$ and O$_2^+$ as reagent ions. For easier ion identification, fastGC was used when available.

The measurements were done over different campaigns in Innsbruck, at IONICON Analytik GmbH. The data analysis was equally split between the three PhD students


\begin{table}[ht]
\centering
\caption{List of analysed ketones.} %(add reference, this was from PubChem).}
\label{tb:k}
\begin{tabular}{llcc}
\toprule
&\quad \textbf{Compound}	 &\textbf{Formula}\quad	&\textbf{Monoisotopic mass (g/mol)} \quad\\ \midrule 
\ldelim\{{13}{20mm}[\parbox{20mm}{linear}]&2-Butanone					&	C$_{4}$H$_{8}$O		&72.107					\\
&2-Pentanone					&	C$_{5}$H$_{10}$O	&86.134					\\
&3-Pentanone					&	C$_{5}$H$_{10}$O	&86.134					\\
&2-Hexanone					&	C$_{6}$H$_{12}$O	&100.161				\\
&3-Hexanone					&	C$_{6}$H$_{12}$O	&100.161				\\
&2-Heptanone					&	C$_{7}$H$_{14}$O	&114.188				\\
&3-Heptanone					&	C$_{7}$H$_{14}$O	&114.188				\\
&4-Heptanone					&	C$_{7}$H$_{14}$O	&114.188				\\
&3-Octanone					&	C$_{8}$H$_{16}$O	&128.215				\\
&2-Nonanone					&	C$_{9}$H$_{18}$O	&142.242				\\
&3-Nonanone					&	C$_{9}$H$_{18}$O	&142.242				\\
&2-Decanone					&	C$_{10}$H$_{20}$O	&156.269				\\
&3-Decanone					&	C$_{10}$H$_{20}$O	&156.269				\\
\addlinespace[0.1cm]
\ldelim\{{1}{20mm}[\parbox{20mm}{cyclic}]&Cyclohexanone				&	C$_{6}$H$_{10}$O	&98.145					\\
\addlinespace[0.1cm]
\ldelim\{{5}{20mm}[\parbox{20mm}{branched}]&3-Methyl-2-butanone			&	C$_{5}$H$_{10}$O	&86.134					\\
&3-Methyl-2-pentanone		&	C$_{6}$H$_{12}$O	&100.161				\\
&2-Methyl-3-pentanone		&	C$_{6}$H$_{12}$O	&100.161				\\
&2-Methyl-3-hexanone			&	C$_{7}$H$_{14}$O	&114.188				\\
&2-Methyl-3-heptanone		&	C$_{8}$H$_{16}$O	&128.215				\\
\bottomrule
\end{tabular}
\end{table}



\subsection{PTR-ToF-MS 8000}
For this study we used a PTR-ToF-MS 8000 manufactured by IONICON Analytik GmbH. This instrument has been described in detail elsewhere \cite{GRAUS20101037}. This instrument has SRI capabilities and also two add-ins were used when needed: LCU and fastGC. 




\subsection{FastGC}
Ask Felix information of this add-in

\subsection{LCU}

\subsection{Samples}



\begin{figure}
\begin{subfloatrow}
\sidesubfloat[2-butanone]{\scalebox{0.7}{\begin{tikzpicture}\chemfig{[:-30]-[::60](=[::60]O)-[::-60]-[::60]}\end{tikzpicture}}
\label{fig:k1}}
\quad
\sidesubfloat[]{\scalebox{0.7}{\begin{tikzpicture}\chemfig{[:-30]-[::60](=[::60]O)-[::-60]-[::60]-[::-60]}\end{tikzpicture}}
\label{fig:k2}}
\quad
\sidesubfloat[]{\scalebox{0.7}{\begin{tikzpicture}\chemfig{[:30]-[::-60]-[::60](=[::60]O)-[::-60]-[::60]}\end{tikzpicture}}
\label{fig:k3}}
\end{subfloatrow}
\bigskip
\begin{subfloatrow}
\sidesubfloat[]{\scalebox{0.7}{\begin{tikzpicture}\chemfig{[:-30]-[::60](=[::60]O)-[::-60]-[::60]-[::-60]-[::60]}\end{tikzpicture}}
\label{fig:k4}}
\quad
\sidesubfloat[]{\scalebox{0.7}{\begin{tikzpicture}\chemfig{[:30]-[::-60]-[::60](=[::60]O)-[::-60]-[::60]-[::-60]}\end{tikzpicture}}
\label{fig:k5}}
\end{subfloatrow}
\bigskip
\begin{subfloatrow}
\sidesubfloat[]{\scalebox{0.7}{\begin{tikzpicture}\chemfig{[:-30]-[::60](=[::60]O)-[::-60]-[::60]-[::-60]-[::60]-[::-60]}\end{tikzpicture}}
\label{fig:k6}}
\quad
\sidesubfloat[]{\scalebox{0.7}{\begin{tikzpicture}\chemfig{[:30]-[::-60]-[::60](=[::60]O)-[::-60]-[::60]-[::-60]-[::60]}\end{tikzpicture}}
\label{fig:k7}}
\end{subfloatrow}
\bigskip
\begin{subfloatrow}
\sidesubfloat[]{\scalebox{0.7}{\begin{tikzpicture}\chemfig{[:-30]-[::60]-[::-60]-[::60](=[::60]O)-[::-60]-[::60]-[::-60]}\end{tikzpicture}}
\label{fig:k8}}
\quad
\sidesubfloat[]{\scalebox{0.7}{\begin{tikzpicture}\chemfig{[:30]-[::-60]-[::60](=[::60]O)-[::-60]-[::60]-[::-60]-[::60]-[::-60]}\end{tikzpicture}}
\label{fig:k9}}
\end{subfloatrow}
\bigskip
\begin{subfloatrow}
\sidesubfloat[]{\scalebox{0.7}{\begin{tikzpicture}\chemfig{[:-30]-[::60](=[::60]O)-[::-60]-[::60]-[::-60]-[::60]-[::-60]-[::60]}\end{tikzpicture}}
\label{fig:k10}}
\quad
\sidesubfloat[]{\scalebox{0.7}{\begin{tikzpicture}\chemfig{[:30]-[::-60]-[::60](=[::60]O)-[::-60]-[::60]-[::-60]-[::60]-[::-60]-[::60]}\end{tikzpicture}}
\label{fig:k11}}
\end{subfloatrow}
\bigskip
\begin{subfloatrow}
\sidesubfloat[]{\scalebox{0.7}{\begin{tikzpicture}\chemfig{[:-30]-[::60](=[::60]O)-[::-60]-[::60]-[::-60]-[::60]-[::-60]-[::60]-[::-60]}\end{tikzpicture}}
\label{fig:k12}}
\quad
\sidesubfloat[]{\scalebox{0.7}{\begin{tikzpicture}\chemfig{[:30]-[::-60]-[::60](=[::60]O)-[::-60]-[::60]-[::-60]-[::60]-[::-60]-[::60]-[::-60]}\end{tikzpicture}}
\label{fig:k13}}
\end{subfloatrow}
\bigskip
\begin{subfloatrow}
\sidesubfloat[]{\scalebox{0.7}{\begin{tikzpicture}
\chemfig{**6(---(=[::-60]O)---)}
\end{tikzpicture}}
\label{fig:k14}}
\quad\quad
\sidesubfloat[]{\scalebox{0.7}{\begin{tikzpicture}\chemfig{[:-30]-[::60](=[::60]O)-[::-60](-[::-60])-[::60]}\end{tikzpicture}}
\label{fig:k15}}
\quad

\sidesubfloat[]{\scalebox{0.7}{\begin{tikzpicture}\chemfig{[:-30]-[::60](=[::60]O)-[::-60](-[::-60])-[::60]-[::-60]}\end{tikzpicture}}
\label{fig:k16}}
\end{subfloatrow}
\bigskip
\begin{subfloatrow}
\sidesubfloat[]{\scalebox{0.7}{\begin{tikzpicture}\chemfig{[:30]-[::-60](-[::-60])-[::60](=[::60]O)-[::-60]-[::60]}\end{tikzpicture}}
\label{fig:k17}}
\quad
\sidesubfloat[]{\scalebox{0.7}{\begin{tikzpicture}\chemfig{[:30]-[::-60](-[::-60])-[::60](=[::60]O)-[::-60]-[::60]-[::-60]}\end{tikzpicture}}
\label{fig:k18}}
\quad
\sidesubfloat[]{\scalebox{0.7}{\begin{tikzpicture}\chemfig{[:30]-[::-60](-[::-60])-[::60](=[::60]O)-[::-60]-[::60]-[::-60]-[::60]}\end{tikzpicture}}
\label{fig:k19}}
\end{subfloatrow}
\bigskip
\caption{Structure of (a) 2-butanone, (b) 2-pentanone, (c) 3-pentanone, (d) 2-hexanone, (e) 3-hexanone, (f) 2-heptanone, (g) 3-heptanone, (h) 4-heptanone, (i) 3-octanone,  (j) 2-nonanone, (k) 3-nonanone, (l) 2-decanone, (m) 3-decanone, (n) cyclohexanone, (o) 3-methyl-2-butanone, (p) 3-methyl-2-pentanone, (q) 2-methyl-3-pentanone, (r) 2-methyl-3-hexanone and (s) 2-methyl-3-heptanone.}
\label{fig:k}
\end{figure}


\subsection{Experimental procedure}






\section{Results and discussion}
\section{Conclusions and further remarks}



















