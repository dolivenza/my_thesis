\chapter{Conclusions and Further Remarks}
\markboth{Conclusions}{}




% from 21.3.19
%David – many thanks for your comments. Yes – I made a mistake in the summary – you are correct. 
%My interpretation of the data is that the BRs are humidity dependent but not the fragmentation pathways – the same ions are seen under dry and humid conditions. But we have already seen that we don’t understand the BRs even under a constant humidity so we cannot expect to understand the effects of changing the humidity – or for that matter changing E/N. 
%Changing the conditions of the hollow cathode – I have no idea – when I last spoke with Chris he suggested it might be worthwhile so you should ask him what he had in mind.
%A different topic. If we have these reactant ions with sufficient energy to fragment ISOF why do we propose more normal energy reactant ions when we study cocaine and the other compounds? My only suggestion is that the highly energetic ions cause massive fragmentation to small ions and are either lost in the noise or ignored or attributed to impurities. Can you have a look at some of the cocaine and analogues spectra and let me know? It’s a problem we have to address before we can publish parallel papers on cocaine and ISOF – and more importantly need to be addressed in your thesis.
%Best wishes - Peter






%\section{Chapter by chapter}
In chapter [] the results [] were presented. 
This concludes....
Further work is needed to elucidate..




























% \blindtext
% \blindtext
% \blindtext


