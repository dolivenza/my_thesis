\chapter{Conclusions and Further Remarks}
\markboth{Conclusions and Further Remarks}{}

The main goal of this thesis was to extend the underpinning knowledge of the ion-molecule processes taking place in \acrshort{scims} techniques, mainly PTR-MS, through both the experimental and theoretical analysis of ion-molecule reactions of molecules of interest with the reagent ions. % and their dependence on parameters like the reduced electric field. %different parameters
%
To accomplish this, several molecules of importance within the fields of homeland security and medicine have been studied to generate a database-like set of results including the product ions and their abundances at different values of the reduced electric field.
%
These substances of interest include  cocaine and other illicit drugs and related compounds; 
phthalic acid and phthalate esters; 
linear, cyclic and branched ketones; 
TNT and related mono- and di- nitrotoluenes;
nitroaniline isomers;
and 
anaesthetics compounds.



The theoretical work in this thesis comprehends DFT calculations to obtain the energetics of the neutral molecules as well as of the proton transfer and fragmentation reactions these molecules undergo.
%
These were performed by Dr Peter Watts using the B3LYP functional with the 6-31+G(d,p) basis set in Gaussian09W and GaussView05 for Windows.
%
The experimental work was mostly done at the  Molecular Physics group laboratory of the School of Physics and Astronomy at the University of Birmingham, Birmingham (UK)
but part of it was done in collaboration with both industrial and academic institutions: 
Kore Technology, Ltd.,  Ely (UK);
IONICON Analytik GmbH, Innsbruck (Austria);
and
the Institute for Breath Research, Leopold-Franzens-Universität Innsbruck, Dornbirn (Austria).
%and
%Institut für Ionenphysik und Angewandte Physik, Universität Innsbruck, Innsbruck (Austria).


\section{Summary of research findings and future work needed}
%%%%%%%%%%%%%%%%%%%%%%%%%%%%%%%%%%%%%%%%%%%%%%%%%%%%%%%%%%%%%%%%%%%%%%%%%%%%%%%%%%
%\textbf{READ CONCLUSIONS FROM EACH CHAPTER AND MAKE A LITTLE SUMMARY HERE}
%%%%%%%%%%%%%%%%%%%%%%%%%%%%%%%%%%%%%%%%%%%%%%%%%%%%%%%%%%%%%%%%%%%%%%%%%%%%%%%%%%%%
\subsection{Theoretical and experimental investigations of cocaine and related compounds
in PTR-MS}

This chapter comprises a PTR-MS and theoretical (DFT) study of the reactions of cocaine and related compounds with (H$_2$O)$_n$H$_3$O$^+$.
%
A thermal desorption unit was employed for the less volatile substances while the investigation of benzoates and isobutyrates was done through headspace analysis.
%
The product ion signal intensities obtained from the reaction of the studied substances were given as a function of the reduced electric and the drift voltage for experiments in two different conditions of humidity (although for some compounds only the drier data was acquired).
%
The dominant reaction channel was non-dissociative proton transfer for cocaine, cocaethylene, methyl ecgonine, ethyl ecgonine, ....

for the whole \textit{E/N} range.







%___________





For the molecules that resemble a moiety of the cocaine molecule (i.e. benzoate and isobutyrate esters) other this was not always the case.


Further work is needed to elucidate ....

Finding the transition state for the loss of benzoic acid and the study of benzoylecgonine, and the other two ecgonines? are future work.


%%%%%%%%%%%%%%%%%%%%%%%%%%%%%%%%%%%%%%%%%%%%%%%%%%%%%%%%%%%%%%%%%%%%%%%%%%%%%%%%%%%%%%%%%%%%%%%%%%%%%%%%%%%%
\vspace{1.5in}\textbf{From the chapter}


%
The observed product ions qualitatively agree with those described in the literature considering that the ions' collisional energies, and hence the degree of fragmentation, can differ from one analytical method to another. 



% GENERAL___________________________________________________________________
The loss of an alcohol (i.e. MeOH, EtOH or iPrOH in each case) molecule  is a common fragmentation pathway observed in all the substances studied in this chapter except the isobutyrate esters.
%
This is supported by the DFT calculations, which reveal that this is a barrierless process once the proton is attached to the %methyl or ethyl 
ester oxygen.
%
On the contrary, DFT reveals that the formation of some of the fragment ions is not thermodynamically allowed and therefore they are formed through field-activated collision-induced dissociation, like protonated benzene and further fragment ions.

There is however  some disagreement between the DFT and the experimental results  for the loss of C$_2$H$_4$ from ethyl benzoate and ethyl isobutyrate, and the loss of C$_3$H$_6$ from isopropyl benzoate.
%
Whilst these reactions are exergonic, they are only observed at certain values of the reduced electric field.
%
These fragmentation pathways must be therefore kinetically rather than thermodynamically driven.
%
The same argument is considered valid to justify the little observed fragmentation with other compounds.
%
% COCAINE___________________________________________________________________________
The energetics for cocaine and methyl ecgonine predicted a higher abundance of fragment ions than what was observed because the proton is mobile between the protonation sites and the formation of the most fragment ions is exergonic.
%
But PTR-MS results show that the dominant ion is the protonated parent parent across the whole \textit{E/N} range, which suggests that the proton goes to the most basic site (i.e. the pyrolidine nitrogen N1) and stays sequestered there.
%
Furthermore, the transition state for the loss of benzoic acid to yield the ion at \textit{m/z} 182 has not been found yet.
%
Although it is not observed with the range of collisional energies available in PTR-MS, this ion becomes more abundant than the protonated parent at high enough collisional energies as reported in the mzCloud database for electrospray ionisation measurements of cocaine \cite{mzcloudCOC}, which confirms the hypothesis of the fragmentation being kinetically rather than thermodynamically driven.

% closing remarks______________________
Cocaine metabolites and other related compounds were included in this study to complement and facilitate the PTR-MS study of cocaine.
%
However, PTR-MS will almost definitely not be used as an analytical tool for the detection of these metabolites and therefore, contrary to  deployed instruments for field campaigns,  PTR-MS instrument was used here as a research tool, not an analytical tool.


%%%%%%%%%%%%%%%%%%%%%%%%%%%%%%%%%%%%%%%%%%%%%%%%%%%%%%%%%%%%%%%%%%%%%%%%%%%%%%%%%%%%%%%%%%%%%%%%%%%%%%%%%%%%
\newpage
\subsection{Enhancement of Compound Selectivity Using a Radio Frequency Ion Funnel Proton Transfer Reaction Mass Spectrometer: Improved Specificity for Explosive Compounds}




%%%%%%%%%%%%%%%%%%%%%%%%%%%%%%%%%%%%%%%%%%%%%%%%%%%%%%%%%%%%%%%%%%%%%%%%%%%%%%%%%%%%%%%%%%%%%%%%%%%%%%%%%%%%
\vspace{1.5in}\textbf{From the chapter}


%%%%%%%%%%%%%%%%%%%%%%%%%%%%%%%%%%%%%%%%%%%%%%%%%%%%%%%%%%%%%%%%%%%%%%%%%%%%%%%%%%%%%%%%%%%%%%%%%%%%%%
\newpage
\subsection{Selective Reagent Ion Mass Spectrometric Investigations of the Nitroanilines}

This study demonstrated the possible selectivity enhancement for the identification of three nitroaniline isomers when using the SRI-MS technique.
%
The product ion distributions arising from the reaction of the nitroaniline isomers with H$_3$O$^+$ and O$_2^+$ as a function of the reduced electric field are illustrated with quantum chemical  (DFT) calculations.

In the proton transfer mode, the protonated parent ion is the dominant ion for the three isomers for a broad reduced electric field range.
%
However, 3-nitroaniline solvates more than the 2- and 4- isomers at low \textit{E/N}, with the MH$^+$.H$_2$O ion reaching ca. 50\% of the PID at 60 Td.
%
This it is supported by the DFT calculations, which predicted that 3-nitroaniline binds water ca. 10 better than 2- and 4-nitroaniline, and allows to differentiate the 3- isomer from the 2- and 4-.
%
Moreover, the parent ion was also the dominant ion observed with the charge transfer mode for the three nitroanilines, but C$_6$H$_6$NO$^+$ becomes the dominant ion for 4-nitroaniline above approximately 190 Td, which allows the identification of the 4- isomer, as this never is higher than ca. 15\% for 2- or 3-nitroaniline.


%%%%%%%%%%%%%%%%%%%%%%%%%%%%%%%%%%%%%%%%%%%%%%%%%%%%%%%%%%%%%%%%%%%%%%%%%%%%%%%%%%%%%%%%%%%%%%%%%%%%%%%%%%%%
\subsection{Compendium of the Reactions of H$_3$O$^+$ With Selected Ketones of Relevance
to Breath Analysis Using Proton Transfer Reaction Mass Spectrometry}

This study comprises a large database of the product ions and their relative intensities as a function of the \textit{E/N} for the reactions of H$_3$O$^+$ and H$_3$O$^+$.(H$_2$O) with a variety of ketones.
%
The observed product ion branching ratios can vary when using different instruments with different configurations and therefore they are only indicative of the reactions and fragmentation pathways taking place in the drift tube.
%
Furthermore, the manipulation of the collisional energy did not enhance the selectivity 
and hence a pre-separation stage, such as the fastGC add-on used in this investigation, is needed to separate isomeric compounds.

Future research lines following this study include the investigation of aldehydes in PTR-MS at different values of the reduced electric field.
%
There are two motivations for this: (i) aldehydes are isomers of ketones, and (ii) both ketones and aldehydes are present in breath samples.
%
However, previous PTR-MS studies have already demonstrated that aldehydes fragment considerably more than ketones, and therefore the contribution to the protonated parent ion signal coming from aldehydes will be a small part of that of the ketones at low and mid values of the \textit{E/N} \cite{buhr2002analysis,schwarz2009determining}.



%%%%%%%%%%%%%%%%%%%%%%%%%%%%%%%%%%%%%%%%%%%%%%%%%%%%%%%%%%%%%%%%%%%%%%%%%%%%%%%%%%%%%%%%%%%%%%%%%%%%%%%%%%%%
\subsection{Investigation of Phthalate Esters in Proton Transfer Reaction Mass Spectrometry
via Direct Headspace Sampling}


These results demonstrate the feasibility of the detection of phthalic acid and ten phthalate esters in PTR-MS by means of headspace analysis. 
%
The product ion distributions arising from the reaction of the phthalates with H$_3$O$^+$ were provided for a function of the reduced electric range from 80 to 205 Td.
%
The identification of three isomers, namely  DBP, DiBP and MEHP, was proved possible without initial pre-separation by solely comparing the product ion distributions at different \textit{E/N} values.
%
Protonated phthalic anhydride (C$_8$H$_5$O$_3^+$, \textit{m/z} 149) is a common product ion observed in this PTR-MS investigation of phthalates, although it was not observed with dimethyl or diallyl phthalate.
%
For the alkyl diester phthalates the abundance of protonated phthalic anhydride  increases with the alkyl chain length and with the reduced electric field.
%
The loss of the corresponding alcohol (i.e. methanol for DMP, ethanol for DEP, propanol for DPP and butanol for DBP) from the protonated parent was also observed for all the studied alkyl diester phthalates.
%
In contrast to \textit{m/z} 149, the abundance of the loss of an alcohol decreases with the alkyl chain length. 







% from 21.3.19
%David – many thanks for your comments. Yes – I made a mistake in the summary – you are correct. 
%My interpretation of the data is that the BRs are humidity dependent but not the fragmentation pathways – the same ions are seen under dry and humid conditions. But we have already seen that we don’t understand the BRs even under a constant humidity so we cannot expect to understand the effects of changing the humidity – or for that matter changing E/N. 
%Changing the conditions of the hollow cathode – I have no idea – when I last spoke with Chris he suggested it might be worthwhile so you should ask him what he had in mind.
%A different topic. If we have these reactant ions with sufficient energy to fragment ISOF why do we propose more normal energy reactant ions when we study cocaine and the other compounds? My only suggestion is that the highly energetic ions cause massive fragmentation to small ions and are either lost in the noise or ignored or attributed to impurities. Can you have a look at some of the cocaine and analogues spectra and let me know? It’s a problem we have to address before we can publish parallel papers on cocaine and ISOF – and more importantly need to be addressed in your thesis.
%Best wishes - Peter





























% \blindtext
% \blindtext
% \blindtext


