\chapter{Conclusions and Further Remarks}
\markboth{Conclusions and Further Remarks}{}

The main goal of this thesis was to extend the underpinning knowledge of the ion-molecule processes taking place in \acrshort{scims} techniques, mainly PTR-MS, through both the experimental and theoretical analysis of ion-molecule reactions of molecules of interest. % and their dependence on parameters like the reduced electric field. %different parameters
%
To accomplish this, several molecules of importance within the fields of homeland security and medicine have been studied to generate a database-like set of results including the product ions and their abundances at different values of the reduced electric field.
%
These substances of interest include  cocaine and other illicit drugs and related compounds; 
phthalic acid and phthalate esters; 
linear, cyclic and branched ketones; 
TNT and related mono- and di- nitrotoluenes;
nitroaniline isomers;
and 
anaesthetics compounds.



The theoretical work in this thesis comprehends DFT calculations to obtain the energetics of the neutral molecules and the proton transfer and fragmentation reactions they undergo.
%
These were performed by Dr Peter Watts using the B3LYP functional with the 6-31+G(d,p) basis set in Gaussian09W and GaussView05 for Windows.
%
The experimental work has been mainly done at the  Molecular Physics group laboratory of the School of Physics and Astronomy at the University of Birmingham, Birmingham (UK)
but part of it was done in collaboration with both industrial and academic institutions: 
Kore Technology, Ltd.,  Ely (UK);
IONICON Analytik GmbH, Innsbruck (Austria);
and
the Institute for Breath Research, Leopold-Franzens-Universität Innsbruck, Dornbirn (Austria).
%and
%Institut für Ionenphysik und Angewandte Physik, Universität Innsbruck, Innsbruck (Austria).


\section{Summary of research findings and future work needed}


\subsection{Theoretical and experimental investigations of cocaine and related compounds
in PTR-MS}


These compounds have been investigated at different collisional energies (i.e. drift voltages and reduced electric fields), using in some cases (or when necessary) a thermal desorption unit to desorb the less volatile compounds.



Previous experimental work with the same TDU is that from \citeauthor{blenkhorn2019novel} and  \citeauthor{gonzalez2017development}
\cite{blenkhorn2019novel,gonzalez2017development}










Finding the transition state for the loss of benzoic acid and the study of benzoylecgonine are future work.






%\subsection{Enhancement of Compound Selectivity Using a Radio Frequency Ion Funnel Proton Transfer Reaction Mass Spectrometer: Improved Specificity for Explosive Compounds}







\subsection{Selective Reagent Ion Mass Spectrometric Investigations of the Nitroanilines}

Three nitroaniline isomers where studied by means of SRI-...........




\subsection{Compendium of the Reactions of H$_3$O$^+$ With Selected Ketones of Relevance
to Breath Analysis Using Proton Transfer Reaction Mass Spectrometry}


FastGC and LCU with a PTR TOF80000000000



\subsection{Investigation of Phthalate Esters in Proton Transfer Reaction Mass Spectrometry
via Direct Headspace Sampling}


Headspace of phthalate esters..........
















% from 21.3.19
%David – many thanks for your comments. Yes – I made a mistake in the summary – you are correct. 
%My interpretation of the data is that the BRs are humidity dependent but not the fragmentation pathways – the same ions are seen under dry and humid conditions. But we have already seen that we don’t understand the BRs even under a constant humidity so we cannot expect to understand the effects of changing the humidity – or for that matter changing E/N. 
%Changing the conditions of the hollow cathode – I have no idea – when I last spoke with Chris he suggested it might be worthwhile so you should ask him what he had in mind.
%A different topic. If we have these reactant ions with sufficient energy to fragment ISOF why do we propose more normal energy reactant ions when we study cocaine and the other compounds? My only suggestion is that the highly energetic ions cause massive fragmentation to small ions and are either lost in the noise or ignored or attributed to impurities. Can you have a look at some of the cocaine and analogues spectra and let me know? It’s a problem we have to address before we can publish parallel papers on cocaine and ISOF – and more importantly need to be addressed in your thesis.
%Best wishes - Peter






%\section{Chapter by chapter}
%In chapter [] the results [] were presented. 
%This concludes....
%Further work is needed to elucidate..




























% \blindtext
% \blindtext
% \blindtext


